%%%%%%%%%%%%%%%%%%%%%%%%%%%%%%%%%%%%%%%%%
% Twenty Seconds Resume/CV
% LaTeX Template
% Version 1.0 (14/7/16)
%
% Original author:
% Carmine Spagnuolo (cspagnuolo@unisa.it) with major modifications by 
% Vel (vel@LaTeXTemplates.com) and Harsh (harsh.gadgil@gmail.com)
%
% License:
% The MIT License (see included LICENSE file)
%
%%%%%%%%%%%%%%%%%%%%%%%%%%%%%%%%%%%%%%%%%

%----------------------------------------------------------------------------------------
% PACKAGES AND OTHER DOCUMENT CONFIGURATIONS
%----------------------------------------------------------------------------------------

\documentclass[letterpaper]{twentysecondcv} % a4paper for A4

% Command for printing skill overview bubbles
% \newcommand\skills{ 
% ~
%   \smartdiagram[bubble diagram]{
%         \textbf{Data}\\\textbf{Engineering},
%         \textbf{Full Stack}\\\textbf{Dev},
%         \textbf{~~~~~~~~OOP~~~~~~~~~},
%         \textbf{~~~~~~Machine~~~~~~}\\\textbf{~~Learning~~},
%         \textbf{~~~~~DevOps~~~~~}
%     }
% }

% Programming skill bars
\programming{{Python $\textbullet$ Scala / 3}, {C $\textbullet$ C++  $\textbullet$ Java $\textbullet$ Kotlin / 4}, {Objective-C $\textbullet$ Swift $\textbullet$ JS $\textbullet$ TypeScript / 6}}

% Projects text
\education{
\textbf{MS, Computer Science} (GPA equiv: 3.8) \\
Specialization: Artificial Intelligence \\
University of Calabria \\
2007 - 2010 | Calabria, Italy

\textbf{BS, Computer Science} (GPA equiv: 4.0) \\
University of Calabria \\
2004 - 2007 | Calabria, Italy
}

% Patents

\patents{
\textbf{Co-inventor of P38048USP1} \\
Accessing Multiple Domains Across Multiple Devices For Candidate Responses. \\
Apple Inc., Mar 2018.

\textbf{Co-inventor of P33966US1} \\
Feedback Analysis of a Digital Assistant. \\
Apple Inc., Aug 2017.

\textbf{Co-inventor of P34063US1} \\
Synchronization and Task Delegation of a Digital Assistant. \\
Apple Inc., Jun 2017.
% \\
% Also filed as P34063DK1 and P34063CN1.

\textbf{Co-inventor of P34373US1} \\
Optimizing Dialogue Policy Decisions for Digital Assistants Using Implicit Feedback. \\
Apple Inc., Jan 2017.
% \\
% Also filed as P34373DK1 and P34373WO1.
}

% Awards

\awards{
\textbf{Summa cum laude} \\
Awarded to the top students in their final year of a Masters degree. \\
University of Calabria, 2010.

% \textbf{Best Bachelor's Thesis} \\
% University of Calabria, 2007.

\textbf{Summa cum laude} \\
Awarded to the top students in their final year of a Bachelor degree. \\
University of Calabria, 2007.
}

%----------------------------------------------------------------------------------------
%  PERSONAL INFORMATION
%----------------------------------------------------------------------------------------
% If you don't need one or more of the below, just remove the content leaving the command, e.g. \cvnumberphone{}

\cvname{Gennaro \vspace{2mm} \newline Frazzingaro} % Your name
\cvjobtitle{ Sr. Engineering Manager } % Job
% title/career

\cvlinkedin{in/gennarofrazzingaro}
\cvgithub{}
\cvnumberphone{+1 (408) 896 2074} % Phone number
% \cvsite{} % Personal website
\cvmail{gennarinoos@gmail.com} % Email address

%----------------------------------------------------------------------------------------

\begin{document}

\makeprofile % Print the sidebar

%----------------------------------------------------------------------------------------
%  EXPERIENCE
%----------------------------------------------------------------------------------------

% \section{Experience}
\vspace{0.3cm}

\begin{twenty} % Environment for a list with descriptions
    \twentyitem
        {March 2019 -}
        {Present}
        {Sr. Engineering Manager, Clinical Health Insights}
        {\href{http://www.apple.com/}{Apple Inc.}}
        {Managing and growing a team of 5 engineers, whose responsibility include:
        {\begin{itemize}
        % \itemsep0em
        \item Developing the platform to download both clinical records from your medical institution, and in-house curated ontology of medical terms derived from SNOMED, LOINC, and others, to the user.
        \item Enabling charting of etherogenosly coded/labeled records together, and the understanding of drug-drug interactions.
        \item Laying the groundwork for on-device ML-based predictions over health data.
        \end{itemize}}
        }
        {}
     \\
    \twentyitem
        {Jul 2017 -}
        {March 2019}
        {Sr. Engineering Manager, Siri Distributed Platform}
        {\href{http://www.apple.com}{Apple Inc.}}
        {Managed a team of 8 engineers, grew the team from 6 to 8, mentored engineers to take leadership roles in key areas, such as:
        {\begin{itemize}
        % \itemsep0em
        \item Supporting on-device execution to enable better performance, privacy, and personalization of Siri responses, without any loss in developer agility.
        \item Improving Siri's performance on slower devices, up to 4s faster.
        \item Making Siri 100\% reliable for a subset of domains, by supporting offline execution.
        % \item Siri's portable runtime, coordinating SiriKit and non-SiriKit integration points both on server and iOS, watchOS, and macOS.
        % \item Enabled further personalization of Siri's understanding and personalized responses, respecting the user's privacy.
        \end{itemize}}
        }
        {}
    \\
    \twentyitem
        {Jan 2015 -}
        {Jul 2017}
        {Sr. Software Engineer, Siri Platform}
        {\href{http://www.apple.com}{Apple Inc.}}
        {Senior staff member of the Siri Platform team. Led the development of several features focused on privacy and personalization, and mentored junior engineers. Led large cross-functional projects across different organizations at Apple.}
        {
        {\begin{itemize}
        \item \textit{Siri Tips}: increased usage of poorly discoverable Siri features (between 5\% and 70\% increase in daily usage), by promoting them alongside related Siri responses.
        \item Built the platform to perform on-device analysis of engagement with Siri's responses, without leaking any information about the user. \textit{Patented}.
        \item Closed the feedback loop for such implicit user corrections to train ML models. Increased overall accuracy for en-US, by amost 10\%.
        \item Led the cross-functional work at Apple to collect user engagement, use feedback to replace manual grading, and promote Reinforcement Learning. \textit{Patented}.
        \item Built the first Siri user profile on-device (shared over iCloud keychain), with the objective to deliver personalized Siri user experiences. \textit{Patented}.
        \item Built a platform for a Siri server to execute arbitrary code on user's devices while idle. \textit{Patented}.
        % \item Member of the Siri Architecture Working Group, and contributed to the offline Siri's architecture definition.
        \end{itemize}}
        }
    \\
    \twentyitem
        {Mar 2012 -}
        {Jan 2015}
        {Software Engineering Manager, Siri Developer Tools}
        {\href{http://www.apple.com}{Apple Inc.}}
        {}
        {
        {\begin{itemize}
        \item Built and led a team of 4 junior engineers, promoted Agile Methodologies both withing the team and in Siri.
        \item Reduced creation of a Siri use case from weeks to less than 24h, through a set of new tools to create Siri's use cases from scratch, and supported the move from rule-based NLP to an ML-based system.
        \item Helped re-define content workflow for designers, ML developers and QA, enabling seemless translation of design into implementation, then into functional testing.
        \item Built tools to allow non-technical content creators to deploy Siri responses to production in less than a day, allowing Marketing, Legal and Design teams to quickly sign-off on what was pushed to production.
        \item Member of the working group in charge of redefining the programming language to code Siri responses, working on the toolset to easily preview them (permutations over a declarative language).
        \end{itemize}}
        }
    \\
  \twentyitem
      {Sep 2009 -}
    {Mar 2012}
        {Software Engineer, Core Localization Technologies}
        {\href{http://www.apple.com}{Apple Inc.}}
        {Responsible for the Italian Localization of the first iPad. Built the toolset to improve the agility of localizing and testing Siri, from 5 languages to 30+ languages.}
        {
        {\begin{itemize}
        \item Built framework for "language generation", exploding GFGs into set of examplars to train statistical models.
        \item Built rule-based system to learn GFGs from POS-tagged exemplars.
        % \item Maintained and improved the Leverage process for Classic Localization, built on top the Apple Translation Memory (TM).
        \item Built a CMS for ”terminology memory”, to define a cross-language glossary of Apple translations, and sistematically leverage previously translation of specific terms across translation units.
        % \item Built various tools to streamline the Localization QA work.
        \end{itemize}}
        }
    \\ 
  %   \twentyitem
  %     {Sep 2009 -}
    % {Jun 2010}
  %       {Localization Engineer \& QA Lead, Localization team}
  %       {\href{http://www.apple.com}{Apple Inc.}}
  %       {}
  %       {
  %       {\begin{itemize}
  %       \item Led the Italian QA efforts for the first Apple iPad.
  %       \item Responsible for automating the testing process of Apple new iOS products.
  %       \end{itemize}}
  %       }
  %    \\
     \twentyitem
      {Feb 2009 -}
    {Jul 2009}
        {Intern, Localization Engineering}
        {\href{http://www.apple.com/}{Apple Inc.}}
        {}
        {
        % \begin{itemize}
        % \item Lead development of an iOS 3 and MacOS application to facilitate and streamline the collection of logs and tracking of issues.
        % \item Built Web applications to streamline the in-country iOS 3 bug tracking and status reporting.
        % \end{itemize}
      }
     \\
     % \twentyitem
     %    {Jan 2008 -}
     %    {Oct 2009}
     %    {Tutor and Laboratory assistant}
     %    {\href{http://www.unical.it/}{University of Calabria}}
     %    {} % {Polyvalent student and teacher assistance for over 12 courses, including Artificial Intelligence, Computability and Complexity and Formal Languages and Compilers.}
     %    {
     %    % \begin{itemize}
     %    % \item Artificial Intelligence,
     %    % \item Computability and Complexity (Theoretical Computer Science),
     %    % \item Object Oriented Programming, Advanced Computer Programming,
     %    % \item Relational Databases, Advanced Databases,
     %    % \item Operating Systems, Computer Networks,
     %    % \item Formal Languages and Compilers,
     %    % \item Introduction to Computer Science, Fundamentals of Computer Programming
     %    % \end{itemize}
     %    }
     % \\
     \twentyitem
        {Mar 2007 -}
        {Sep 2007}
        {Intern Researcher}
        {\href{http://www.deri.ie/}{DERI Galway, Ireland}}
        {Member of the Semantic Web research group, working on a query solver for the Semantic Web using using non-monotonic reasoning over RDF resources.}
        {
        % \begin{itemize}
        % \item Added support query modifiers (DISTINCT, FILTER and others) to the dlvhex-sparql solver.
        % \item Deployed a Semantic Web Service for machine-to-machine interoperability over SPARQL SELECT and CONSTRUCT queries.
        % \item Transformed dlvhex-sparql \href{http://wiki.ruleml.org/index.php/RuleML_Home}{RuleML} output to the W3C standard \href{https://www.w3.org/TR/rdf-sparql-XMLres/}{SPARQL Query Results XML Format}.
        % \end{itemize}
        }
        
  % \twentyitem{<dates>}{<title>}{<location>}{<description>}
\end{twenty}

%----------------------------------------------------------------------------------------
%  EDUCATION
%----------------------------------------------------------------------------------------
% \section{Education}
% \begin{twenty}
%     \twentyitem
%         {Oct 2007 -}
%         {Oct 2010}
%         {Master Degree in Artificial Intelligence}
%         {\href{http://www.unical.it/}{University of Calabria}}
%         {}
%         {
%         {\begin{itemize}
%         \item \textbf{GPA equivalent}: 3.9
%         \item \textbf{Thesis}: Crossing the Line between Procedural and Declarative Programming
%         % \item \textbf{Supervisors}: Wolfgang Faber, Giovambattista Ianni
%         \item Theoretical work on bringing the power of logic based programming paradigm for solving problems in a fully declarative way to procedural languages.
%         \item Implementation of a C++ dylib to showcase performance and applications.
%         % \item \textbf{Tags}: C++, Answer Set Programming, Parsers, Compilers \vspace{2mm}
%         \end{itemize}}
%         }
%   \twentyitem
%       {Oct 2004 -}
%         {Oct 2007}
%         {Bachelor “Laurea” Degree in Computer Science}
%         {\href{http://www.unical.it/}{University of Calabria}}
%         {}
%         {
%         {\begin{itemize}
%         \item \textbf{GPA equivalent}: 4.0
%         \item \textbf{Thesis}: Implementing and Extending SPARQL queries over dlvhex
%         % \item \textbf{Supervisors}: Axel Florian Polleres, Giovambattista Ianni
%         \item Theoretical work on Semantic Web and Web Services in relation with Answer Set Programming.
%         \item Implementation of a SPARQL query solver using logic programming reasoning techniques to retrieve results from RDF datasources.
%         % \item \textbf{Tags}: C++, Java, Semantic Web, Answer Set Programming, RDF, SPARQL, SOAP, WSDL \vspace{2mm}
%     \end{itemize}}
%         }
% \end{twenty}

%----------------------------------------------------------------------------------------
%    PATENTS
%----------------------------------------------------------------------------------------
% \section{Patents}
% \begin{twenty}
%     \twentyitem
%         {Mar 2018}
%         {}
%         {\small{Co-inventor of P38048USP1}}
%         {Apple Inc.}
%         {}
%         {Accessing Multiple Domains Across Multiple Devices For Candidate Responses. \vspace{2mm}}
%     \twentyitem
%         {Aug 2017}
%         {}
%         {\small{Co-inventor of P33966US1}}
%         {Apple Inc.}
%         {}
%         {Feedback Analysis of a Digital Assistant. \vspace{2mm}}
%     \twentyitem
%         {Jun 2017}
%         {}
%         {\small{Co-inventor of P34063US1, P34063DK1, and P34063CN1}}
%         {Apple Inc.}
%         {}
%         {Synchronization and Task Delegation of a Digital Assistant. \vspace{2mm}}
%     \twentyitem
%         {Jan 2017}
%         {}
%         {\small{Co-inventor of P34373US1, P34373DK1, and P34373WO1}}
%         {Apple Inc.}
%         {}
%         {Optimizing Dialogue Policy Decisions for Digital Assistants Using Implicit Feedback. \vspace{2mm}}
% \end{twenty}

%----------------------------------------------------------------------------------------
%    AWARDS
%----------------------------------------------------------------------------------------
% \section{Awards}
% \begin{twenty}
%     \twentyitem
%         {Oct 2010}
%         {}
%         {Summa cum laude}
%         {Master in Artificial Intelligence, University of Calabria}
%         {}
%         {Awarded to the top student in their final year of a Masters degree. \vspace{2mm}}
%     \twentyitem
%         {Jul 2007}
%         {}
%         {Summa cum laude}
%         {Master in Artificial Intelligence, University of Calabria}
%         {}
%         {Awarded to the top student in their final year of a Bachelor degree.}
% \end{twenty}

\end{document} 
