%%%%%%%%%%%%%%%%%%%%%%%%%%%%%%%%%%%%%%%%%
% Twenty Seconds Resume/CV
% LaTeX Template
% Version 1.0 (14/7/16)
%
% Original author:
% Carmine Spagnuolo (cspagnuolo@unisa.it) with major modifications by 
% Vel (vel@LaTeXTemplates.com) and Harsh (harsh.gadgil@gmail.com)
%
% License:
% The MIT License (see included LICENSE file)
%
%%%%%%%%%%%%%%%%%%%%%%%%%%%%%%%%%%%%%%%%%

%----------------------------------------------------------------------------------------
%	PACKAGES AND OTHER DOCUMENT CONFIGURATIONS
%----------------------------------------------------------------------------------------

\documentclass[letterpaper]{twentysecondcv} % a4paper for A4

% Command for printing skill overview bubbles
% \newcommand\skills{ 
% ~
% 	\smartdiagram[bubble diagram]{
%         \textbf{Data}\\\textbf{Engineering},
%         \textbf{Full Stack}\\\textbf{Dev},
%         \textbf{~~~~~~~~OOP~~~~~~~~~},
%         \textbf{~~~~~~Machine~~~~~~}\\\textbf{~~Learning~~},
%         \textbf{~~~~~DevOps~~~~~}
%     }
% }

% Programming skill bars
\programming{{Python $\textbullet$ Scala $\textbullet$ Perl / 3}, {C $\textbullet$ C++  $\textbullet$ Java $\textbullet$ Scripting / 5.5}, {Objective-C $\textbullet$ Swift $\textbullet$ JS $\textbullet$ TypeScript / 6}}

% Projects text
\education{
\textbf{MS, Computer Science} (GPA equiv: 3.8) \\
Specialization: Artificial Intelligence \\
University of Calabria \\
2007 - 2010 | Calabria, Italy

\textbf{BS, Computer Science} (GPA equiv: 4.0) \\
University of Calabria \\
2004 - 2007 | Calabria, Italy
}

% Patents

\patents{
\textbf{Co-inventor of P38048USP1} \\
Accessing Multiple Domains Across Multiple Devices For Candidate Responses. \\
Apple Inc., Mar 2018.

\textbf{Co-inventor of P33966US1} \\
Feedback Analysis of a Digital Assistant. \\
Apple Inc., Aug 2017.

\textbf{Co-inventor of P34063US1} \\
Synchronization and Task Delegation of a Digital Assistant. \\
Apple Inc., Jun 2017.
% \\
% Also filed as P34063DK1 and P34063CN1.

\textbf{Co-inventor of P34373US1} \\
Optimizing Dialogue Policy Decisions for Digital Assistants Using Implicit Feedback. \\
Apple Inc., Jan 2017.
% \\
% Also filed as P34373DK1 and P34373WO1.
}

% Awards

\awards{
\textbf{Summa cum laude} \\
Awarded to the top students in their final year of a Masters degree. \\
University of Calabria, 2010.

% \textbf{Best Bachelor's Thesis} \\
% University of Calabria, 2007.

\textbf{Summa cum laude} \\
Awarded to the top students in their final year of a Bachelor degree. \\
University of Calabria, 2007.
}

%----------------------------------------------------------------------------------------
%	 PERSONAL INFORMATION
%----------------------------------------------------------------------------------------
% If you don't need one or more of the below, just remove the content leaving the command, e.g. \cvnumberphone{}

\cvname{Gennaro \vspace{2mm} \newline Frazzingaro} % Your name
\cvjobtitle{ Sr. Engineering Manager } % Job
% title/career

\cvlinkedin{in/gennarofrazzingaro}
\cvgithub{}
\cvnumberphone{+1 (408) 806 2074} % Phone number
% \cvsite{} % Personal website
\cvmail{gennarinoos@gmail.com} % Email address

%----------------------------------------------------------------------------------------

\begin{document}

\makeprofile % Print the sidebar

%----------------------------------------------------------------------------------------
%	 EXPERIENCE
%----------------------------------------------------------------------------------------

% \section{Experience}
\vspace{0.3cm}

\begin{twenty} % Environment for a list with descriptions
    \twentyitem
        {Jul 2017 -}
        {Present}
        {Sr. Engineering Technical Manager, Siri}
        {\href{http://www.apple.com}{Apple Inc.}}
        {Managing a team of 12 developers, primarily responsible for:
        {\begin{itemize}
        \itemsep0em
        \item Distributed execution of a Siri request using portable code, and the platform for Siri ML models to run inference on-device,
        \item Supporting Siri fully offline, to enable further personalization of the user experience whilst respecting the user privacy, and improving Siri's performance and reliability,
        \item Providing the runtime environment for other platforms higher in the stack, such as the publicly known SiriKit and Siri Shortcuts.
        \end{itemize}}
        }
        {}
    \\
    \twentyitem
        {Jan 2015 -}
        {Jul 2017}
        {Sr. Software Engineer / Technology Lead, Siri}
        {\href{http://www.apple.com}{Apple Inc.}}
        {Member of both Server and Client Siri Platform teams, delivering features focused on Privacy and Personalization, covering the full stack, and coding in different programming languages on a daily basis. I've mentored many junior engineers, and worked cross-functionally across teams and organizations.}
        {
        {\begin{itemize}
        \item Built a platform and led a team to collect implicit feedback from Siri users, both during and after a Siri request, and perform on-device analysis, respecting the user’s privacy. Patented.
        \item Led the cross-functional work at Apple to collect user engagement, use feedback to replace manual grading, and promote Reinforcement Learning. Patented.
        \item Built the first Siri user profile on-device, in order to deliver personalized Siri user experiences, and allow Siri itself to promote features to users when appropriate. Data is shared securely over iCloud, across all devices owned by the same user. Patented.
        \item Built a platform to execute arbitrary code on iOS devices during idle time, updatable out of band. Patented.
        \item Member of the Siri Architecture Working Group, and contributed to make Siri available offline.
        \end{itemize}}
        }
    \\
    \twentyitem
        {Mar 2012 -}
        {Jan 2015}
        {Sr. Tools Engineering / Technical Manager, Siri}
        {\href{http://www.apple.com}{Apple Inc.}}
        {}
        {
        {\begin{itemize}
        \item Built and led a team of 4 developers, and promoted Agile Methodologies, including CI, CD, and TDD.
        \item Built a content management system as a web tool for Siri use cases, bridging the gap between design, implementation and testing, and contributed in redefining the process for creating a Siri use case. The CMS uses technologies such as Typescript, AngularJS, MongoDB, Solr, and integrates with the GitHub APIs.
        \item Coordinated cross-functional work across Siri Design, Operations, Security, Machine Learning, Siri Build, Siri ASR and NLP, and Siri QA.
        \item Led the efforts to improve the XML-based programming language used to define Siri's responses.
        \item Built tools to explode permutations of Siri's response XML representation, in order to ease the review process of Siri's 
        utterances by MarCom, Design and Apple Legal.
        \item Built tools to allow non-technical content creators to deploy Siri responses to production.
        \end{itemize}}
        }
    \\
	\twentyitem
    	{Sep 2009 -}
		{Mar 2012}
        {Software Engineer, Core Localization Technologies}
        {\href{http://www.apple.com}{Apple Inc.}}
        {}
        {
        {\begin{itemize}
        \item Member of the team responsible to build the tools to streamline the process of localizing Siri in 4+ languages after the acqui-hire, and build the foundation that allows Siri to speak 21+ languages (and counting) today.
        \item Built a framework for Language Generation, to transform Context Free Grammars (GFGs) into examples suitable to train statistical models.
        \item Built a rule-based system to learn Context Free Grammars (GFGs) from POS-tagged exemplars.
        % \item Maintained and improved the Leverage process for Classic Localization, built on top the Apple Translation Memory (TM).
        \item Built content management system for ”terminology memory”, to define a cross-language glossary of Apple translations, and sistematically leverage previously translation of specific terms across translation units.
        % \item Built various tools to streamline the Localization QA work.
        \end{itemize}}
        }
    \\ 
  %   \twentyitem
  %  		{Sep 2009 -}
		% {Jun 2010}
  %       {Localization Engineer \& QA Lead, Localization team}
  %       {\href{http://www.apple.com}{Apple Inc.}}
  %       {}
  %       {
  %       {\begin{itemize}
  %       \item Led the Italian QA efforts for the first Apple iPad.
  %       \item Responsible for automating the testing process of Apple new iOS products.
  %       \end{itemize}}
  %       }
  %    \\
     \twentyitem
   		{Feb 2009 -}
		{Jul 2009}
        {Intern, Localization Engineering}
        {\href{http://www.apple.com/}{Apple Inc.}}
        {}
        {
        \begin{itemize}
        \item Lead development of an internal-only iOS applications to facilitate the creation of issues and collection of logs from the device.
        \item Built Web applications to streamline the in-country issue tracking and reporting of the iPhone International QA team.
        \end{itemize}
    	}
     \\
     % \twentyitem
     %    {Jan 2008 -}
     %    {Oct 2009}
     %    {Tutor and Laboratory assistant}
     %    {\href{http://www.unical.it/}{University of Calabria}}
     %    {Polyvalent student and teacher assistance for over 12 courses, including Artificial Intelligence, Computability and Complexity and Formal Languages and Compilers.}
     %    {
     %    % \begin{itemize}
     %    % \item Artificial Intelligence,
     %    % \item Computability and Complexity (Theoretical Computer Science),
     %    % \item Object Oriented Programming, Advanced Computer Programming,
     %    % \item Relational Databases, Advanced Databases,
     %    % \item Operating Systems, Computer Networks,
     %    % \item Formal Languages and Compilers,
     %    % \item Introduction to Computer Science, Fundamentals of Computer Programming
     %    % \end{itemize}
     %    }
     % \\
     \twentyitem
        {Mar 2007 -}
        {Sep 2007}
        {Intern Researcher}
        {\href{http://www.deri.ie/}{DERI Galway, Ireland}}
        {Member of the Semantic Web research group, working on a query solver for the Semantic Web using using non-monotonic reasoning over RDF resources.}
        {
        % \begin{itemize}
        % \item Added support query modifiers (DISTINCT, FILTER and others) to the dlvhex-sparql solver.
        % \item Deployed a Semantic Web Service for machine-to-machine interoperability over SPARQL SELECT and CONSTRUCT queries.
        % \item Transformed dlvhex-sparql \href{http://wiki.ruleml.org/index.php/RuleML_Home}{RuleML} output to the W3C standard \href{https://www.w3.org/TR/rdf-sparql-XMLres/}{SPARQL Query Results XML Format}.
        % \end{itemize}
        }
        
	% \twentyitem{<dates>}{<title>}{<location>}{<description>}
\end{twenty}

%----------------------------------------------------------------------------------------
%	 EDUCATION
%----------------------------------------------------------------------------------------
% \section{Education}
% \begin{twenty}
%     \twentyitem
%         {Oct 2007 -}
%         {Oct 2010}
%         {Master Degree in Artificial Intelligence}
%         {\href{http://www.unical.it/}{University of Calabria}}
%         {}
%         {
%         {\begin{itemize}
%         \item \textbf{GPA equivalent}: 3.9
%         \item \textbf{Thesis}: Crossing the Line between Procedural and Declarative Programming
%         % \item \textbf{Supervisors}: Wolfgang Faber, Giovambattista Ianni
%         \item Theoretical work on bringing the power of logic based programming paradigm for solving problems in a fully declarative way to procedural languages.
%         \item Implementation of a C++ dylib to showcase performance and applications.
%         % \item \textbf{Tags}: C++, Answer Set Programming, Parsers, Compilers \vspace{2mm}
%         \end{itemize}}
%         }
% 	\twentyitem
%     	{Oct 2004 -}
%         {Oct 2007}
%         {Bachelor “Laurea” Degree in Computer Science}
%         {\href{http://www.unical.it/}{University of Calabria}}
%         {}
%         {
%         {\begin{itemize}
%         \item \textbf{GPA equivalent}: 4.0
%         \item \textbf{Thesis}: Implementing and Extending SPARQL queries over dlvhex
%         % \item \textbf{Supervisors}: Axel Florian Polleres, Giovambattista Ianni
%         \item Theoretical work on Semantic Web and Web Services in relation with Answer Set Programming.
%         \item Implementation of a SPARQL query solver using logic programming reasoning techniques to retrieve results from RDF datasources.
%         % \item \textbf{Tags}: C++, Java, Semantic Web, Answer Set Programming, RDF, SPARQL, SOAP, WSDL \vspace{2mm}
% 		\end{itemize}}
%         }
% \end{twenty}

%----------------------------------------------------------------------------------------
%    PATENTS
%----------------------------------------------------------------------------------------
% \section{Patents}
% \begin{twenty}
%     \twentyitem
%         {Mar 2018}
%         {}
%         {\small{Co-inventor of P38048USP1}}
%         {Apple Inc.}
%         {}
%         {Accessing Multiple Domains Across Multiple Devices For Candidate Responses. \vspace{2mm}}
%     \twentyitem
%         {Aug 2017}
%         {}
%         {\small{Co-inventor of P33966US1}}
%         {Apple Inc.}
%         {}
%         {Feedback Analysis of a Digital Assistant. \vspace{2mm}}
%     \twentyitem
%         {Jun 2017}
%         {}
%         {\small{Co-inventor of P34063US1, P34063DK1, and P34063CN1}}
%         {Apple Inc.}
%         {}
%         {Synchronization and Task Delegation of a Digital Assistant. \vspace{2mm}}
%     \twentyitem
%         {Jan 2017}
%         {}
%         {\small{Co-inventor of P34373US1, P34373DK1, and P34373WO1}}
%         {Apple Inc.}
%         {}
%         {Optimizing Dialogue Policy Decisions for Digital Assistants Using Implicit Feedback. \vspace{2mm}}
% \end{twenty}

%----------------------------------------------------------------------------------------
%    AWARDS
%----------------------------------------------------------------------------------------
% \section{Awards}
% \begin{twenty}
%     \twentyitem
%         {Oct 2010}
%         {}
%         {Summa cum laude}
%         {Master in Artificial Intelligence, University of Calabria}
%         {}
%         {Awarded to the top student in their final year of a Masters degree. \vspace{2mm}}
%     \twentyitem
%         {Jul 2007}
%         {}
%         {Summa cum laude}
%         {Master in Artificial Intelligence, University of Calabria}
%         {}
%         {Awarded to the top student in their final year of a Bachelor degree.}
% \end{twenty}

\end{document} 
